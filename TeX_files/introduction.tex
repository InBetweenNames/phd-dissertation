\documentclass[../main.tex]{subfiles}

\begin{document}

%TODO: chapter 0
\chapter{Preface}
\begin{refsection}
    My involvement in the SpeechWeb project began in 2009 when I had begun my Undergraduate
    degree in Computer Science at the University of Windsor.  Back then, I worked with Dr. Frost as part of my Outstanding Scholars scholarship and became familiar with developing SpeechWeb applications in the Haskell programming language.

    Dr. Frost and Dr. Rahmatullah Hafiz had successfully shown in 2012 that it was possible to parse highly ambiguous left-recursive context-free grammars using functional combinators as part of Dr. Hafiz's doctoral dissertation. This opened the doors to a larger NSERC project
    that was intended to demonstrate that Compositional Semantics is an appropriate choice
    for building a Natural Language Query Interface to the Semantic Web.

    In 2013, I published my Undergraduate thesis with my colleagues at the ASONAM'13 conference \cite{donais2013system}.

    In 2014, although I wasn't formally credited, I helped contribute to a paper that was published at the ESWC \cite{frost2014demonstration}.

    In 2016, I completed my Master's degree with Dr. Frost as my supervisor.  My task was to extend the English coverage of the NLQI with prepositional phrases.  In doing so, I discovered that the word ``by'' as in ``discovered by'' could be treated in the same way as a preposition, simplifying the grammar and the semantics while also allowing more flexible queries.

    In 2017, I began my Doctoral studies aiming at bringing the SpeechWeb to the Semantic Web.
    This would require both improving the time complexity of query evaluation within the semantics, and also enhancing the query interface with an even broader coverage of English.  It would also require allowing the NLQI to run on a wide variety of devices, including those with low power requirements and using the semantics with non-event based triplestores.  Throughout the course of my studies, Dr. Frost and I published
    papers.  My contributions to those papers are as follows: %TODO

    I explored multiple avenues towards solving the problem.  I learned about heterogeneous computing and became involved with the Khronos Group OpenCL Working Group with Paul Preney at the University of Windsor.  We published a poster paper at IWOCL 2017 that dealt with build systems for OpenCL applications.  We were both credited in the OpenCL 2.2 specification released that year.
    %TODO: add citation?

    In 2017, Eric Matthews, a fellow graduate student at the University of Windsor, thanked me in his Master's Thesis for helping him write his Thesis Report.

    At the end of 2017, I obtained an OCE TalentEdge Academic Internship for developing a simulator for Additive Manufacturing processes.  I published a paper on leveraging heterogeneous computing for accelerating simulations for these applications in 2018 at CAD Conference in Paris.  At that conference, I was invited to submit a full journal paper which was published in 2019. %TODO: make sure both are referenced

    I began looking for ways to accelerate the SpeechWeb project using Heterogeneous Programming.
    Using the knowledge I obtained from low-level microarchitectural programming, I was able to successfully get the Haskell demo from my Master's thesis running on a low level consumer network router and I realized that the work would be appropriate for Internet of Things (IoT) applications.

    In 2018, Dr. Frost and I published our first paper together at NLIWoD, a satellite event of the ISWC, seeking to re-awaken interest in Compositional Semantics as an approach for creating Natural Language Query Interfaces.  In 2019, we published at WEBIST and were invited to submit to the WEBIST Springer Journal.  In 2020, we published at IEEE ICSC and again at WEBIST.

    In 2020, I successfully ported the application over to WebAssembly such that it could be run directly in web browsers.

    As of October 2020, my Master's thesis has been cited 4 times according to Google Scholar and my 2013 ASONAM paper has been cited 8 times according to Microsoft Academic.

    % * State number of citations for each paper


%TODO explain:
% * How I became involved in the project
% * What Richard and Rahmatullah had done up until my involvement
% * What my task was:
%   * Speed up and extend English coverage (with prepositional phrases)
% * Explain what Richard's NSERC was intended to achieve:
%   * Speed up and prove compositionality is appropriate for building an NL interface to the Semantic Web
% * Explain the different stages of the work
% * Explain the papers we produced
% * Explain my contributions to those papers
% * State my thesis statement here and how the papers prove it
% * Change formatting to be consistent across all papers
% * Explain how the result of the work is more general, and the approach could be used to solve similar problems...
% This is all to help the readers understand my contributions
\end{refsection}

\chapter{Introduction}
\begin{refsection}

\label{chapter:intro}

%INTRODUCTION:

\section{Motivation}

%TODO: Add more here?  Tie in Semantic Web with Natural Language, and talk about how NLP interfaces could be used to control IoT devices

\section {The Problem}

\begin{itemize}
    \item NLQIs to the Semantic Web
    \item Internet of Things
    \item Accessibility
\end{itemize}


%SUMMARY OF EXISTING APPROACHES:
\section {Existing approaches}


%THE PROBLEM WITH EXISTING APPROACHES:

\section {Shortcomings of previous approaches}


\section {New approach}
%OVERVIEW OF NEW APPROACH:


\section{Thesis Statement}

This dissertation contains five papers that address the problem of creating Natural Language Query Interfaces to the Semantic Web.  It proves the following statement:

\textit{``A scalable, efficient, expressive and precise method for processing natural-language queries to the Semantic Web can be built using a Compositional Semantics.''}

This dissertation describes research that proves the thesis. The research was originally published in refereed conference and journal papers that are summarized in Chapters \ref{chapter:nliwod2018conf} through \ref{chapter:webist2020conf}.

The rest of this chapter introduces the papers above with related background information and the contributions that I made to them. %TODO: should that be in preface?

%TODO
Please see \hyperref[appendix:b]{Appendix B} for a list of all papers related to this Thesis that I authored or co-authored. \cite{peelar2016accommodating}

\section{Proof of Concept}

%TODO
Rather than prove the thesis statement mathematically, I have chosen to build an interface that
proves that the thesis is correct.

\section{Notation}

In addition to standard Lambda Calculus and set notation, the following notation is used in this dissertation:
\begin{itemize}
    \item $\entity{x}$ denotes the entity $x$
    \item $\event{1234}$ denotes the event \#1234
    \item $\meaningof{x}$ is the mathematical denotation of the phrase $x$
    \item $\wordset{x}$ is the set of all entities that are members of $x$
    \item $\wordpred{x}$ the logical predicate associated with the word $x$
    \item $\wordfdbr{x}$ is the FDBR of all entities that are members of $x$ (a datastructure first introduced in \Cref{chapter:nliwod2018conf})
    \item Queries are written in ``\texttt{monospaced}'' font
    \item \property{property} is used to denote a property, role, or type of an event.
    \item \textit{name} is used to indicate query results and objects in the real world.
\end{itemize}
For example, $\entity{phobos}$ is the mathematical object representing \textit{phobos}, a moon that orbits Mars.

%\section{Structure of Dissertation}
%
%The remainder of this Dissertation is structured as follows:

%\begin{enumerate}
%	\item An Extensible Natural-Language Query Interface to an Event-Based Semantic Web Triplestore
%	\item A New Data Structure for Processing Natural Language Database Queries
%	\item A Compositional Semantics for a Wide-coverage Natural-Language Query Interface to a Semantic Web Triplestore
%	\item Accepted: A New Approach for Processing Natural-Language Queries to Semantic Web Triplestores (Journal)
%    \item Accepted for publication: Accommodating Negation in an Efficient Event-Based Natural Language Query Interface to the Semantic Web (WEBIST 2020)
%\end{enumerate}

\section{An Extensible Natural-Language Query Interface to an Event-Based Semantic Web Triplestore}

In this paper we discuss:

\begin{itemize}
    \item the Function Defined by a Binary Relation (FDBR)
\end{itemize}

\Cref{chapter:nliwod2018conf}

\section{A New Data Structure for Processing Natural Language Database Queries}

\Cref{chapter:webist2019conf}

This paper was published at WEBIST 2019.  We were invited to submit an extended paper to the WEBIST Springer Journal, and at the conference we were nominated for Best Student Paper Award.  In this paper we discuss:

\begin{itemize}
    \item How the FDBR is used to create denotations of other linguistic constructs such as superlatives
    \item How our semantics can be readily adapted to relational databases
\end{itemize}

\section{A Compositional Semantics for a Wide-coverage Natural-Language Query Interface to a Semantic Web Triplestore}

This conference paper was published at IEEE ICSC 2020.  In this paper we discuss:

\begin{itemize}
    \item The $n^2 - n$ functions defined by an $n$-ary relation
    \item Accommodating transitive verbs with $n$-ary relations
\end{itemize}

\Cref{chapter:icsc2020conf}

\section{A New Approach for Processing Natural-Language Queries to Semantic Web Triplestores}

\Cref{chapter:webist2019journal}

We were invited to submit this paper to the WEBIST Springer Journal and it has been accepted for publication.

In this paper we discuss:

\begin{itemize}
    \item A full treatment of how to adapt our semantics to relational databases
    \item How to improve the computational complexity of the semantics through memoization
    \item Strategies for handling both syntactic and semantic ambiguity in returned results
\end{itemize}

\section{Accommodating Negation in an Efficient Event-Based Natural Language Query Interface to the Semantic Web}

\Cref{chapter:webist2020conf}

This conference paper was accepted for presentation at WEBIST 2020.

In this paper we discuss:

\begin{itemize}
    \item How to handle negation in the semantics
\end{itemize}

%TODO: notation

%\pagebreak
%TODO: these need to be numbered from 1 each time
\printbibliography[heading=subbibintoc]
\end{refsection}

\end{document}