\documentclass[../main.tex]{subfiles}

\begin{document}

\chapter{Introduction}

\label{chapter:intro}

%INTRODUCTION:

\section{Motivation}

%TODO: Add more here?  Tie in Semantic Web with Natural Language, and talk about how NLP interfaces could be used to control IoT devices

\section {The Problem}




%SUMMARY OF EXISTING APPROACHES:
\section {Existing approaches}


%THE PROBLEM WITH EXISTING APPROACHES:

\section {Shortcomings of previous approaches}


\section {New approach}
%OVERVIEW OF NEW APPROACH:


\section{Thesis Statement}

This dissertation contains four papers that address...

It proves the following statement:

\textit{``A scalable, efficient, expressive and precise method for processing natural-language queries to the Semantic Web can be built using a Compositional Semantics.''}

The rest of this introduction will introduce the papers...

Please see \Cref{appendix} for a list of all papers related to this Thesis that I authored or co-authored.

\section{Proof of Concept} 

\section{Structure of Dissertation}

The remainder of this Dissertation is structured as follows:

\begin{enumerate}
	\item An Extensible Natural-Language Query Interface to an Event-Based Semantic Web Triplestore
	\item A New Data Structure for Processing Natural Language Database Queries 
	\item A Compositional Semantics for a Wide-coverage Natural-Language Query Interface to a Semantic Web Triplestore
	\item Submitted: A New Approach for Processing Natural-Language Queries to Semantic Web Triplestores (Journal)
\end{enumerate}

\section{An Extensible Natural-Language Query Interface to an Event-Based Semantic Web Triplestore}

\section{A New Data Structure for Processing Natural Language Database Queries}

\section{A Compositional Semantics for a Wide-coverage Natural-Language Query Interface to a Semantic Web Triplestore}

\section{A New Approach for Processing Natural-Language Queries to Semantic Web Triplestores}

Submitted to Journal

\end{document}