\documentclass[../main.tex]{subfiles}

\begin{document}

%TODO: chapter 0
\chapter{Preface}
\begin{refsection}
    My involvement in the SpeechWeb project began in 2009 when I had begun my Undergraduate
    degree in Computer Science at the University of Windsor.  Back then, I worked with Dr. Frost as part of my Outstanding Scholars scholarship and became familiar with developing SpeechWeb applications in the Haskell programming language.

    Dr. Frost and Dr. Rahmatullah Hafiz had successfully shown in 2012 that it was possible to parse highly ambiguous left-recursive context-free grammars using functional combinators as part of Dr. Hafiz's doctoral dissertation. This opened the doors to a larger NSERC project
    that was intended to demonstrate that Compositional Semantics is an appropriate choice
    for building a Natural Language Query Interface to the Semantic Web.  At the same time, I set out
    to work on my Undergraduate thesis, and in 2013, it was presented at ASONAM'13 \cite{donais2013system}.

    In 2014, although I wasn't formally credited, I helped contribute to a paper that was published at the ESWC \cite{frost2014demonstration}.  I developed the query program that Dr. Frost used in his demonstration at the conference.  The main reason I wasn't credited was that I joined the research group after the paper itself had been submitted.

    In 2016, I completed my Master's degree with Dr. Frost as my supervisor.  My task was to extend the English coverage of the NLQI with chained prepositional phrases.  In doing so, I discovered that the word ``by'' as in ``discovered by'' could be treated in the same way as a preposition, simplifying the grammar and the semantics while also allowing for more flexible queries.

    In 2017, I began my Doctoral studies aiming at bringing the SpeechWeb to the Semantic Web.
    This would require both improving the time complexity of query evaluation within the semantics, and also enhancing the query interface with an even broader coverage of English.  It would also require allowing the NLQI to run on a wide variety of devices, including those with low power requirements and using the semantics with non-event based triplestores.  During this year, Eric Matthews, a fellow graduate student at the University of Windsor, completed his Master's degree thanked me in his Thesis Report for helping him write his thesis \cite{matthews2017passive}.

    I explored multiple avenues towards solving the problem.  I learned about heterogeneous computing and became involved with the Khronos Group OpenCL Working Group with my colleague Paul Preney at the University of Windsor.  We published a poster paper at IWOCL 2017 that dealt with extending build systems for heterogeneous OpenCL applications.  We were both credited in the OpenCL 2.2 specification released that year \cite{khronos2018opencl}.

    Pursuing heterogeneous computing, at the end of 2017, I obtained an OCE TalentEdge Academic Internship to develop a simulator for Additive Manufacturing processes.  I published a paper on leveraging heterogeneous computing for accelerating simulations for these applications in 2018 at CAD Conference in Paris \cite{peelar2018toolpath}.  There, I was invited to submit a full paper to the CAD and Applications Journal which was subsequently published in 2019 \cite{peelar2019real}.  The work is featured in the software APlus which is used by companies around the world for Additive Manufacturing applications.

    Although the mathematics behind developing 3D Printing simulations and Natural Language Processing applications are dissimilar, both benefit from the same High Performance Computing techniques for implementing and accelerating them.  In particular, I learned a lot about making programs efficient at the microarchitectural level using techniques such as SIMD vectorization and using efficient data layouts. Although none of the papers in this dissertation use heterogeneous computing, notes are made in the Future Work sections of each paper presented where specialized hardware could be used to accelerate certain computations required by the semantics.  In particular, FPGAs, a form of reprogrammable hardware, could be used to accelerate the construction of the FDBR, a datastructure central to the thesis.

    Using the microarchitectural-level insight I had gained from this research track, I was able to successfully get the Haskell demo from my Master's thesis running on a low power consumer network router and I realized that the work would be appropriate for Internet of Things (IoT) applications.
    As we pursued this avenue in the course of my studies, Dr. Frost and I published in both conferences and book series.  My contributions to those papers are described as follows:

    In 2018, Dr. Frost and I published our first paper in this line of research together at NLIWoD \cite{frost2018extensible}, a satellite event of the ISWC, seeking to re-awaken interest in Compositional Semantics as an approach for creating Natural Language Query Interfaces.  We also discussed how certain superlatives and graded quantifiers could be handled within the semantics.  My contributions to this paper included the implementation of the demonstration program, including the website, as well as the discussion of the FDBR, a fundamental datastructure used in this thesis, and how prepositional phrases are handled.

    In 2019, we presented at WEBIST \cite{frostpeelar2019}, where our paper was nominated for the Best Student Paper Award.  Our paper dealt with how our semantics can accommodate ``non-compositional'' features of English such as superlatives.  In particular, we give mention to the $n^2 - n$ binary relations that can be obtained from $n$-ary events such as those that underlie $n$-ary transitive verbs.  We were subsequently invited to submit an extended paper as a chapter for the WEBIST Springer Book.
    My contribution to this paper included the discussion of how our approach could be used with traditional relational databases, the demonstration program and implementation, and some of the examples and discussion around them.

    In February 2020, we presented at IEEE ICSC \cite{peelar2020compositional}.  In our IEEE ICSC 2020 paper, we described how to handle transitive verbs with $n$-ary relations in the semantics and gave a full treatment of the semantics in the Lambda Calculus.  We also had the beginnings of an idea how to adapt the semantics to relational databases.  My contributions to this paper include the discussion of how to generate FDBRs from $n$-ary relations, the formal denotations for chained prepositional phrases, as well as parts of the denotation used for transitive verbs and the examples shown. We were invited to submit to the ASTESJ journal for a special issue, but we did not submit a paper.

    In March 2020, we submitted our WEBIST Springer Book \cite{peelar2020webistjournal} paper dealing with improving the computational complexity of our approach by showing how a Compositional Semantics can be memoized.  A complete application architecture is presented that permits both online and offline computation of the FDBR datastructures that are fundamental to the semantics.  We also show how superlative phrases such as ``the most'' can be accommodated.  My contributions to this paper were all aspects related to memoization, implementation, demonstration, discussion of syntactic and semantic ambiguity, description of the application architecture, denotation for the superlative phrase ``the most'' including nesting superlatives phrases within chained prepositional phrases, discussion regarding accommodating negation, and the discussion for how to use the approach with relational databases.  As of November 2020, our paper is now available via Springer.

    In November 2020, we presented our latest paper at WEBIST \cite{peelarfrostwebist2020}.  Our WEBIST 2020 paper describes how to accommodate negation in our semantics for applications where the Closed World Assumption holds.  Denotations for words entailing negation, such as ``not'', ``non'' and ``no'', are presented along with a denotation for transitive verbs that can handle negated expressions.  Notably, the denotations for ``no'', ``non'' and ``not'' can be omitted to restore the Open World Assumption where appropriate.
    My contributions to this paper include the modifications to the semantics to accommodate negation, including integrating negation with the architecture presented in the WEBIST Springer Book chapter, the denotation for ``not'', and the example queries given.

    As of October 2020, we have successfully embedded the semantics directly within the web browser to remove the need of intermediary servers to process queries.  The goal of this is to allow your web browser to communicate directly with Semantic Web triplestores as though they were ordinary websites served over HTTP.  Currently, a version of the query interface is available that does this with event-based triplestores.  We aim to publish our approach in a functional programming conference as we believe it to be a useful method to create Natural Language Query Interfaces directly in the web browser.

    We have also gained industry interest in crossing the gap to non event-based triplestores using Machine Learning approaches \cite{timbr}.  Our next goal is to combine the techniques developed above to query DBPedia with our semantics by leveraging a Machine Learning approach to perform reification on the non-event based triples.  This will allow us to directly benchmark our approach against other query interfaces.

    As of December 2020, my Master's thesis \cite{peelar2016accommodating} has been cited 4 times according to Google Scholar and my ASONAM'13 paper has been cited 8 times according to Microsoft Academic.

    % * State number of citations for each paper


%TODO explain:
% * How I became involved in the project
% * What Richard and Rahmatullah had done up until my involvement
% * What my task was:
%   * Speed up and extend English coverage (with prepositional phrases)
% * Explain what Richard's NSERC was intended to achieve:
%   * Speed up and prove compositionality is appropriate for building an NL interface to the Semantic Web
% * Explain the different stages of the work
% * Explain the papers we produced
% * Explain my contributions to those papers
% * State my thesis statement here and how the papers prove it
% * Explain how the result of the work is more general, and the approach could be used to solve similar problems...
\printbibliography[heading=subbibintoc]
\end{refsection}

\chapter{Introduction}
\begin{refsection}

\label{chapter:intro}

%INTRODUCTION:

The Internet of Things (IoT) is an emerging phenomenon in the public space.  Remotely controlled lights,
voice assistants connected directly to powerful search engines, and refrigerators that can predict
when you'll run out of food are just some examples of where IoT is entering the lives of people around the world.  Users with accessibility needs could especially benefit from these ``smart'' devices if they were able to interact with them through speech.

Privacy is a particular concern about popular voice assistants that are currently in use.
They function by sending their queries directly to a remote server for processing which then
return results to the user.  This architecture makes their use in confidential environments problematic,
such as in a medical or law office.

Another concern is about the trustworthiness of the returned results.  In expert systems,
it's not enough to just have the answer to a query -- users need to be confident that the
result is indeed correct.

This thesis describes a framework for building NLQIs using Compositional Semantics that addresses these concerns.  In particular, the use of a Compositional Semantics
guarantees that the results returned will be as correct as the data in the database.
It is also auditable: information is available for the returned results that justifies their inclusion in the result.  This can be used to verify that the retrieved information indeed exists within the triplestore or database it was retrieved from.

The NLQIs produced with this framework are able to be run directly on the user's own devices,
such as their own computer, smartphone, or even internet router.  The actual queries the user
makes never leave the device -- the only time information leaves the device is when a query to
a Semantic Web triplestore needs to be made to satisfy it.  For domain-specific applications
where confidentiality is important, the server can be maintained entirely within that environment
under the user's control.

Our approach relies on a memoized event-based Compositional Semantics (CS) supporting complex linguistic structures such as prepositional phrases, superlatives, and negation.  It is \textit{scalable}, in that
the architecture can scale from both small to very large triplestores.  It is \textit{efficient}, in that
it can run on low power hardware.  It is \textit{expressive}, capable of handling queries with many complex linguistic features, and finally it is \textit{precise}, as it is based on a Compositional Semantics,
where the answers returned are as correct as the information retrieved from the triplestore.


%\section{Motivation}

%\section {The Problem}
%
%\begin{itemize}
%    \item NLQIs to the Semantic Web
%    \item Internet of Things
%    \item Accessibility
%\end{itemize}


%SUMMARY OF EXISTING APPROACHES:
%\section {Existing approaches}


%THE PROBLEM WITH EXISTING APPROACHES:

%\section {Shortcomings of previous approaches}


%\section {New approach}
%OVERVIEW OF NEW APPROACH:


\section{Thesis Statement}

This dissertation contains five papers that address the problem of creating Natural Language Query Interfaces to the Semantic Web.  It proves the following statement:

\vspace{0.5em}

\textit{``A scalable, efficient, expressive and precise method for processing natural-language queries to the Semantic Web can be built using a Compositional Semantics.''}

\vspace{0.5em}

This dissertation describes research that proves the thesis. The research was originally published in refereed conference papers and book chapters that are reproduced in Chapters \ref{chapter:nliwod2018conf} through \ref{chapter:webist2020conf}. The rest of this chapter introduces the papers above along with their novel contributions and related background information.

Please see \hyperref[appendix:b]{Appendix B} for a list of all 10 papers related to this Thesis that I authored or co-authored.

\section{Proof of Concept}

Rather than prove the thesis statement mathematically, I have chosen to build an interface that
proves that the thesis is correct.  It can be found at the following URL:

\begin{center}
    \url{https://speechweb2.cs.uwindsor.ca/solarman/demo_sparql.html}
\end{center}
Additionally, a version that runs entirely within the web browser is available at
the following URL:
\begin{center}
    \url{https://speechweb2.cs.uwindsor.ca/solarman-wasm/}
\end{center}
A list of example queries is provided on both websites and in the papers presented in this report.

\subsection{Speech support}
Both interfaces are speech enabled -- simply click on the microphone icon to the left of the query box and speak a question into your microphone.  You may need to give Solarman permission to access your microphone -- if this is the case, click ``Allow'' on the prompt that appears. After speaking,
the system will automatically perform the query and use synthesized speech to read the result aloud
back to you.  For example, try pressing the microphone button and speaking ``\texttt{who discovered a moon that orbits mars}'' into your microphone -- you should hear back the answer ``\texttt{hall}'' from the NLQI.  Supporting web browsers include Mozilla Firefox and Google Chrome-based browsers at this time, including Microsoft Edge.  Internet Explorer is not supported.

\section{Novel Contributions}

We have made several novel contributions while researching the Thesis Statement.  They are outlined below:
\begin{itemize}
    \item A denotation for $n$-ary transitive verbs in a CS
    \item A denotation for superlatives and comparatives such as ``most'' and ``the most''
    \item How the FDBR datastructure can be used to accommodate queries with ``non-compositional'' features in a CS for a NLQI
    \item A memoized query evaluation framework for efficiently evaluating NL queries to a triplestore using a CS
    \item A mapping between event-based triplestores and relational databases, and how our event semantics can handle both types of databases
\end{itemize}
These contributions are described in more detail in the following chapters of this dissertation.

\section{Limitations}

Our approach is currently geared towards expert systems and domain-specific applications. Although it
maintains a very wide coverage of English in queries, it is intended to be used in curated knowledge bases where there is a high degree of certainty about the correctness of the contained information.  One property of our approach is that the answer is as correct as what is contained within the databases themselves.  Our approach has not yet been formally evaluated against other NLQIs -- we provide only qualitative comparisons with other NLQIs in this dissertation. In our approach, a URI identifies an entity or an event uniquely within the universe of discourse.  In this thesis we have used URIs which identify entities uniquely within this Windsor project.

Currently, numerical quantifiers such as ``one'', ``two'' depend upon the Single Role Assumption, where an event may have at most one entity fulfilling a particular role.

Note that while our demonstration uses our ``Solarman'' knowledge base to answer NL queries about the solar system, the semantics as presented are highly general and could be adapted to many different types of knowledge bases.  One would need to understand the domain-specific aspects of those knowledge bases and provide a vocabulary to facilitate this.  For example, with respect to a medical knowledge base, one could answer the query ``Does $X$ contraindicate $Y$?'', where $X$ and $Y$ are names of drugs, by using our denotation for transitive verbs as the denotation of ``contraindicate'' and selecting the relation underlying that verb as appropriate.  This applies to all other syntactic categories as well.

\section{An Extensible Natural-Language Query Interface to an Event-Based Semantic Web Triplestore}

\textbf{\Cref{chapter:nliwod2018conf}} contains a paper that describes the Function Defined by a Binary Relation (FDBR) and how it can be used to answer Natural Language queries to event-based Semantic Web Triplestores.  It was presented at NLIWOD 2018, a satellite event of the ISWC.

Briefly, Semantic Web Triplestores are databases that contain \textit{Resource Description Framework (RDF) triples} that describe facts about \textit{entities}.  Each component of a triple is a Uniform Resource Identifier (URI) that uniquely identifies an entity within that triplestore.  As an example for how these are used, consider the following:

\begin{code}
    <hall> <discover> <phobos> .
    <hall> <discover> <deimos> .
\end{code}

The triples above can be read as ``\textit{subject-predicate-object}''.

Triplestores are accessible via a query method using an endpoint.  The most common query method in use for Semantic Web triplestores is \textit{SPARQL} \cite{sparql}, however, increasingly users are turning to other query methods as well, such as Linked Data Fragments \cite{verborgh2014web} being the recommended choice for querying DBPedia \cite{dbpedia}.

One problem with entity-based triples such as the example above is that it is difficult to add contextual information to a set of triples.  In the example above, it is not clear if Asaph Hall discovered both Phobos and Deimos at the same time or if these were separate events, and it is not clear what year those discoveries took place.  The triples must be \textit{reified} \cite{antoniou2004semantic} to obtain the contextual information.

One method for reification is to use \textit{event-based triplestores}, where the \textit{subject} of each triple identifies an \textit{event} rather than an entity.  The example above could be expressed as:

\begin{code}
    <event1045> <type> <discovery> .
    <event1045> <subject> <hall> .
    <event1045> <object> <phobos> .
    <event1045> <year> <1877> .
    <event1046> <type> <discovery> .
    <event1046> <subject> <hall> .
    <event1046> <object> <deimos> .
    <event1046> <year> <1877> .
\end{code}

Event-based triplestores allow additional contextual information about an event to be expressed
expressed by simply adding additional triples with that event as the subject.  In the example
above, it is clear that both discovery events for Deimos and Phobos took place within the same year.

We present a Natural Language Query Interface to event-based triplestores using the Function Defined
by a Binary Relation (FDBR), a datastructure first used by Frost in \cite{frost1989constructing}
to provide a denotation for binary transitive verbs in a set-theoretic version of Montague Semantics \cite{Dowty:wall}.  In 2016, Peelar showed that the FDBR can be used to accommodate chained
prepositional phrases in a Compositional Semantics \cite{peelar2016accommodating}.  In this paper,
we discuss how the FDBR can be used to answer other types of Natural Language queries as well, including
those with superlatives and graded quantifiers.

\section{A New Data Structure for Processing Natural Language Database Queries}

\textbf{\Cref{chapter:webist2019conf}} contains a paper that describes how the FDBR is used
to create denotations of other linguistic constructs such as superlatives.  It also discusses
how the semantics can be readily adapted to relational databases.  This paper was published at WEBIST 2019.  We were invited to submit an extended paper to the WEBIST Springer Book, and at the conference we were nominated for Best Student Paper Award.

\section{A Compositional Semantics for a Wide-coverage Natural-Language Query Interface to a Semantic Web Triplestore}

\textbf{\Cref{chapter:icsc2020conf}} contains a paper that describes the $n^2 - n$ functions defined by an $n$-ary relation and how these can be used to accommodate transitive verbs in the semantics with $n$-ary relations.  This paper was presented at IEEE ICSC 2020.  We were invited to submit to the ASTESJ journal for a special issue, but we did not submit a paper.

\section{A New Approach for Processing Natural-Language Queries to Semantic Web Triplestores}

\textbf{\Cref{chapter:webist2019journal}} contains a paper that describes how to adapt our semantics
to relational databases.  It provides a full treatment of the application architecture of our query interface, and discusses strategies for handling both syntactic and semantic ambiguity in the returned results over both speech and text modalities.  It also describes how the semantics are memoized to improve their computational complexity and discusses that framework enables FDBRs to be precomputed and cached offline for fast retrieval.

One of the key ideas behind this paper is to memoize the semantics by using the expression tree obtained from the query itself.  Each expression is uniquely named and the results are cached from the triplestore, drastically reducing the number of evaluations that need to be performed during a query.  The same architecture can be used to cache and generate FDBRs offline for fast retrieval later.  It addresses both the efficiency and scalability aspects of the thesis statement.

We were invited to submit this paper for inclusion in the WEBIST Springer Book, where it has been published.

\section{Accommodating Negation in an Efficient Event-Based Natural Language Query Interface to the Semantic Web}

\textbf{\Cref{chapter:webist2020conf}} contains a paper that describes how to accommodate negation
within the semantics.  This conference paper was presented at WEBIST 2020.
The key idea behind this paper is to track cardinality throughout the semantics by introducing
a new triplestore querying primitive to obtain the cardinality of the triplestore.  Negation
in general only holds if the \textit{Closed World Assumption} can be satisfied.  Informally, the Closed World Assumption can be characterized by the statement:

{\hfil ``\textit{The absence of evidence can be assumed as being evidence of absence}''. \hfil} \\
For example, if a particular entity $p$ is not explicitly stated as being a member of the ``person'' set, then it can be assumed that $p$ is not a member of that set.  The \textit{Open World Assumption} on the other hand does not assume this statement to be true.  In the previous example, this would mean that $p$ cannot be assumed as not being a member of the ``person'' set unless there is an explicit statement of non-membership elsewhere in the database.

RDF itself is built on the Open World Assumption, however certain domain-specific triplestores may have enough information such that the Closed World Assumption is valid for that triplestore.  Where it is not, the denotations for ``not'', ``non'', and ``no'' can be omitted, restoring the Open World Assumption
that underlies the Semantic Web.  This flexibility allows our approach to be used in expert systems and domain-specific applications where either assumption is appropriate.

\section{Notation}

In addition to standard Lambda Calculus and set-theory notation, the following notation is used in this dissertation:
\begin{itemize}
    \item ``\texttt{<subject> <predicate> <object> .}'' denotes an RDF triple
    \item $\entity{x}$ denotes the entity $x$
    \item $\event{1234}$ denotes the event \#1234
    \item $\meaningof{x}$ is the mathematical denotation of the phrase $x$
    \item $\wordset{x}$ is the set of all entities that are members of $x$
    \item $\wordpred{x}$ the logical predicate associated with the word $x$
    \item $\wordfdbr{x}$ is the FDBR of all entities that are members of $x$ (a datastructure first introduced in \Cref{chapter:nliwod2018conf})
    \item Queries are written in a \texttt{monospaced} font
    \item \property{property} is used to denote a property, role, or type of an event.
    \item \textit{name} is used to indicate query results and objects in the real world.
\end{itemize}
For example, $\entity{phobos}$ is the mathematical object representing \textit{phobos}, a moon that orbits Mars.

%\pagebreak
\printbibliography[heading=subbibintoc]
\end{refsection}

\end{document}