\documentclass[../main.tex]{subfiles}

\begin{document}

%TODO: chapter 0
\chapter{Preface}
\begin{refsection}
    My involvement in the SpeechWeb project began in 2009 when I had begun my Undergraduate
    degree in Computer Science at the University of Windsor.  Back then, I worked with Dr. Frost as part of my Outstanding Scholars scholarship and became familiar with developing SpeechWeb applications in Haskell.

    Dr. Frost and Dr. Rahmatullah Hafiz had successfully shown in 2012 that it was possible to parse highly ambiguous left-recursive context-free grammars using functional combinators as part of Dr. Hafiz's doctoral dissertation. This opened the doors to a larger NSERC project
    that was intended to demonstrate that Compositional Semantics is an appropriate choice
    for building a Natural Language Query Interface to the Semantic Web.

    In 2014, although I wasn't formally credited, I helped contribute to a paper that was published at the ESWC \cite{frost2014demonstration}.


    In 2016, I completed my Master's degree with Dr. Frost as my supervisor.  My task was to extend the English coverage of the NLQI with prepositional phrases.  In doing so, I discovered that the word ``by'' as in ``discovered by'' could be treated in the same way as a preposition, simplifying the grammar and the semantics while also allowing more flexible queries.

    In 2017, I began my Doctoral studies aiming at bringing the SpeechWeb to the Semantic Web.
    This would require both improving the time complexity of query evaluation within the semantics, and also enhancing the query interface with an even broader coverage of English.  It would also require allowing the NLQI to run on a wide variety of devices, including those with low power requirements and using the semantics with non-event based triplestores.  Throughout the course of my studies, Dr. Frost and I published %TODO
    papers.  My contributions to those papers are as follows:



%TODO explain:
% * How I became involved in the project
% * What Richard and Rahmatullah had done up until my involvement
% * What my task was:
%   * Speed up and extend English coverage (with prepositional phrases)
% * Explain what Richard's NSERC was intended to achieve:
%   * Speed up and prove compositionality is appropriate for building an NL interface to the Semantic Web
% * Explain the different stages of the work
% * Explain the papers we produced
% * Explain my contributions to those papers
% * State my thesis statement here and how the papers prove it
% * Change formatting to be consistent across all papers
% * Explain how the result of the work is more general, and the approach could be used to solve similar problems...
% This is all to help the readers understand my contributions
\end{refsection}

\chapter{Introduction}
\begin{refsection}

\label{chapter:intro}

%INTRODUCTION:

\section{Motivation}

%TODO: Add more here?  Tie in Semantic Web with Natural Language, and talk about how NLP interfaces could be used to control IoT devices

% * State that Eric Matthews thanked me for helping him with his Thesis
% * State number of citations for each paper
% * Move acronyms to a separate page in the Thesis (NLQI, MS, ML, etc)

\section {The Problem}




%SUMMARY OF EXISTING APPROACHES:
\section {Existing approaches}


%THE PROBLEM WITH EXISTING APPROACHES:

\section {Shortcomings of previous approaches}


\section {New approach}
%OVERVIEW OF NEW APPROACH:


\section{Thesis Statement}

This dissertation contains five papers that address...

This dissertation proves the following statement:

\textit{``A scalable, efficient, expressive and precise method for processing natural-language queries to the Semantic Web can be built using a Compositional Semantics.''}

This dissertation describes research that proves the thesis. The research was originally published in refereed conference and journal papers that are summarized in Chapters \ref{chapter:nliwod2018conf} through \ref{chapter:webist2020conf}.

The rest of this introduction will introduce the papers...

%TODO
Please see \hyperref[appendix:b]{Appendix B} for a list of all papers related to this Thesis that I authored or co-authored. \cite{peelar2016accommodating}

\section{Proof of Concept}

%TODO
Rather than prove the thesis statement mathematically, I have chosen to build an interface that
proves that the thesis is correct.

\section{Structure of Dissertation}

The remainder of this Dissertation is structured as follows:

%\begin{enumerate}
%	\item An Extensible Natural-Language Query Interface to an Event-Based Semantic Web Triplestore
%	\item A New Data Structure for Processing Natural Language Database Queries
%	\item A Compositional Semantics for a Wide-coverage Natural-Language Query Interface to a Semantic Web Triplestore
%	\item Accepted: A New Approach for Processing Natural-Language Queries to Semantic Web Triplestores (Journal)
%    \item Accepted for publication: Accommodating Negation in an Efficient Event-Based Natural Language Query Interface to the Semantic Web (WEBIST 2020)
%\end{enumerate}

\section{An Extensible Natural-Language Query Interface to an Event-Based Semantic Web Triplestore}

\Cref{chapter:nliwod2018conf}

\section{A New Data Structure for Processing Natural Language Database Queries}

\Cref{chapter:webist2019conf}

Invited to submit to journal, nominated for Best Student Paper Award

\section{A Compositional Semantics for a Wide-coverage Natural-Language Query Interface to a Semantic Web Triplestore}

\Cref{chapter:icsc2020conf}

\section{A New Approach for Processing Natural-Language Queries to Semantic Web Triplestores}

\Cref{chapter:webist2019journal}

Accepted to WEBIST Journal

\section{Accommodating Negation in an Efficient Event-Based Natural Language Query Interface to the Semantic Web}

\Cref{chapter:webist2020conf}

Accepted at WEBIST 2020

%\pagebreak
%TODO: these need to be numbered from 1 each time
\printbibliography[heading=subbibintoc]
\end{refsection}

\end{document}