% !TeX document-id = {9edb2af6-76b5-485f-a607-d45f08e1925c}
\documentclass[fleqn, oneside, 12pt]{book}

% !TEX program = lualatex
% !BIB program = biber
% !TEX encoding = UTF-8 Unicode

%\usepackage[OT1]{fontenc}
%\usepackage{fontspec}
%\usepackage[T1]{fontenc}

%\setsansfont{CMU Sans Serif}%{Arial}

%\setmainfont{Liberation Serif}%{Times New Roman}

%\setmonofont{CMU Typewriter Text}%{Consolas}
\usepackage[top=1in, bottom=1in, left=1.5in, right=1in]{geometry}
\usepackage{subfiles}
\usepackage[nottoc,numbib,notlof,notlot]{tocbibind}
\usepackage{listings}
\usepackage{amsthm}
\usepackage{thmtools}
\usepackage{amssymb,amsmath}
\usepackage{url}
\usepackage{verbatim}
\usepackage{graphicx}
\usepackage{tabularx}
\usepackage{mathptmx}% http://ctan.org/pkg/mathptmx
\usepackage{fancyhdr}
\usepackage{setspace}
%\usepackage[mark]{gitinfo2}
\usepackage[hidelinks]{hyperref}
\usepackage[backend=biber,bibstyle=numeric-comp,sorting=ydnt,natbib=true,mcite=true,maxnames=100,url=true,isbn=false,doi=false,uniquename=init,giveninits=true,hyperref=true,backref=true,date=edtf,sortcites]{biblatex}
\usepackage{tocloft}
\usepackage{minted}
\usepackage{cleveref}

%for NLIWoD2018 conference paper
\usepackage{enumitem}

%for WEBIST2019 conference paper
\usepackage{booktabs}
\usepackage[olditem,oldenum]{paralist}

%for IEEE ICSC 2020 conference paper
\usepackage[bottom]{footmisc}

\newcommand{\listappendicesname}{List of Appendices}
\newlistof{appendices}{apc}{\listappendicesname}

\newcommand{\listabbreviationsname}{List of Abbreviations}
\newlistof{abbreviations}{abc}{\listabbreviationsname}

\newcommand{\listnewtablesname}{List of Tables}
\newlistof[chapter]{newtables}{ntb}{\listnewtablesname}

%For abbreviation spacing in List of Abbreviations
\newcommand{\abbreviation}[2]{\addcontentsline{abc}{section}{\protect\makebox[3em][l]{\thesection}\protect\makebox[5em][l]{\textbf{#1}} {#2}}}

%\usepackage{underscore}
\graphicspath{ {images/} }

\addbibresource{frostpaper.bib}

\newtheoremstyle{definitionsty}{3pt}{3pt}{\slshape}{}{\bfseries}{.}{.5em}{}
\theoremstyle{definitionsty}
\newtheorem{tdefn}{Definition}[chapter]
\newenvironment{defn}
{\begin{shaded}\begin{tdefn}}
		{\end{tdefn}\end{shaded}}

\usepackage{etoolbox}
\makeatletter
\patchcmd\thmtlo@chaptervspacehack
{\addtocontents{loe}{\protect\addvspace{10\p@}}}
{\addtocontents{loe}{\protect\thmlopatch@endchapter\protect\thmlopatch@chapter{\thechapter}}}
{}{}
\AtEndDocument{\addtocontents{loe}{\protect\thmlopatch@endchapter}}
\long\def\thmlopatch@chapter#1#2\thmlopatch@endchapter{%
	\setbox\z@=\vbox{#2}%
	\ifdim\ht\z@>\z@
	\hbox{\bfseries\chaptername\ #1}\nobreak
	#2
	\addvspace{10\p@}
	\fi
}
\def\thmlopatch@endchapter{}

\makeatother
\renewcommand{\thmtformatoptarg}[1]{ -- #1}
\renewcommand{\listtheoremname}{Nomenclature}

\fancypagestyle{revision}{
	\fancyhf{}
	\fancyhead[L]{\nouppercase{\leftmark}}
	\fancyhead[R]{\thepage}
	%\fancyfoot[L]{Revision 3}
}

\pagestyle{revision}

\newtheorem{definition}{Definition}
\newtheorem{theorem}{Theorem}[section]
\newtheorem{lemma}{Lemma}[theorem]
\newtheorem{proposition}{Proposition}[section]
\newtheorem{subproposition}{Proposition}[proposition]
\newtheorem{corollary}{Corollary}[theorem]

\lstloadlanguages{Haskell}

\lstnewenvironment{code}
{\lstset{}%
    \csname lst@SetFirstLabel\endcsname}
{\csname lst@SaveFirstLabel\endcsname}
\lstset{
    basicstyle={\ttfamily},
    flexiblecolumns=false,
    basewidth={0.5em,0.5em},
    literate={+}{{$+$}}1 {/}{{$/$}}1 {*}{{$*$}}1 {=}{{$=$}}1 {|}{{$\mid$}}1
    {>}{{$>$}}1 {<}{{$<$}}1 {\\}{{$\lambda$}}1
    {\\\\}{{\char`\\\char`\\}}1
    {->}{{$\rightarrow$}}2 {>=}{{$\geq$}}2 {<-}{{$\leftarrow$}}2
    {<=}{{$\leq$}}2 {=>}{{$\implies$}}2
    {\ .\ }{{$\circ$}}2
    {>>}{{>>}}2
    {*>}{{*>}}2
    {>>=}{{>>=}}3
    {>|<}{{>|<}}3
    {<*>}3
    {<+>}3
    {<<*>>}5
}

%\lstnewenvironment{spec}
%{\lstset{}%
%	\csname lst@SetFirstLabel\endcsname}
%{\csname lst@SaveFirstLabel\endcsname}
%\lstset{
%	basicstyle={\ttfamily},
%	flexiblecolumns=false,
%	basewidth={0.5em,0.45em},
%	literate={+}{{$+$}}1 {/}{{$/$}}1 {*}{{$*$}}1 {=}{{$=$}}1
%	{>}{{$>$}}1 {<}{{$<$}}1 {\\}{{$\lambda$}}1
%	{\\\\}{{\char`\\\char`\\}}1
%	{->}{{$\rightarrow$}}2 {>=}{{$\geq$}}2 {<-}{{$\leftarrow$}}2
%	{<=}{{$\leq$}}2 {=>}{{$\Rightarrow$}}2
%	{\ .\ }{{$\circ$}}2
%	{>>}{{>>}}2 {>>=}{{>>=}}2
%	{|}{{$\mid$}}1
%}

\newcommand{\uwinverytightsinglespacelen}{0.9}
\newcommand{\uwintightsinglespacelen}{1.0}
\newcommand{\uwinsinglespacelen}{1.1}
\newcommand{\uwinonehalfspacelen}{1.5}
\newcommand{\uwindoublespacelen}{2.0}
\newcommand{\uwinlistofspacelen}{1.5}
%TODO: spacing %\newcommand{\uwindefaultspacelen}{\uwindoublespacelen}
\newcommand{\uwindefaultspacelen}{\uwinonehalfspacelen}

\newcommand{\uwinverytightsinglespace}%
{\linespread{\uwinverytightsinglespacelen}}
\newcommand{\uwintightsinglespace}%
{\linespread{\uwintightsinglespacelen}}
\newcommand{\uwinsinglespace}%
{\linespread{\uwinsinglespacelen}}
\newcommand{\uwinonehalfspace}%
{\linespread{\uwinonehalfspacelen}}
\newcommand{\uwindoublespace}%
{\linespread{\uwindoublespacelen}}
\newcommand{\uwinlistofspace}%
{\linespread{\uwinlistofspacelen}}
\newcommand{\uwindefaultspace}%
{\linespread{\uwindefaultspacelen}}

\newenvironment{uwinverytightsinglespaceenv}%
{\begin{spacing}{\uwinverytightsinglespacelen}}%
	{\end{spacing}}
\newenvironment{uwintightsinglespaceenv}%
{\begin{spacing}{\uwintightsinglespacelen}}%
	{\end{spacing}}
\newenvironment{uwinsinglespaceenv}%
{\begin{spacing}{\uwinsinglespacelen}}%
	{\end{spacing}}
\newenvironment{uwinonehalfspaceenv}%
{\begin{spacing}{\uwinonehalfspacelen}}%
	{\end{spacing}}
\newenvironment{uwindoublespaceenv}%
{\begin{spacing}{\uwindoublespacelen}}%
	{\end{spacing}}
\newenvironment{uwinlistofspaceenv}%
{\begin{spacing}{\uwinlistofspacelen}}%
	{\end{spacing}}
\newenvironment{uwindefaultspaceenv}%
{\begin{spacing}{\uwindefaultspacelen}}%
	{\end{spacing}}

\long\def\ignore#1{}

\usepackage{fontspec}
\setsansfont{CMU Sans Serif}%{Arial}

\setmainfont{CMU Serif}%{Times New Roman}

\setmonofont{CMU Typewriter Text}%{Consolas}

\newcommand{\dmathit}[1]{\textit{#1}}
\newcommand{\dmathrm}[1]{\text{#1}}
\newcommand{\dmathbf}[1]{\textbf{#1}}

%Macros for denotations
\DeclareMathOperator{\phobos}{phobos}
\DeclareMathOperator{\deimos}{deimos}
\DeclareMathOperator{\every}{every}
\DeclareMathOperator{\spins}{spins}
\DeclareMathOperator{\spin}{spin}
\DeclareMathOperator{\moon}{moon}
\DeclareMathOperator{\moons}{moons}

\newcommand{\meaningof}[1]{\lVert\dmathit{#1\hspace{0.1em}}\rVert} %this *is* text
\newcommand{\eventassoc}[1]{\operatorname{ev}_{#1}}
\newcommand{\event}[1]{\eventassoc{#1}}
\newcommand{\entityassoc}[1]{\mathrm{e}_{\operatorname{#1}}}
\newcommand{\entity}[1]{\entityassoc{#1}}
\newcommand{\wordpred}[1]{\operatorname{#1}_{\operatorname{\mathit{pred}}}}
\newcommand{\wordset}[1]{\operatorname{#1}_{\operatorname{\mathit{set}}}}
\newcommand{\wordfdbr}[1]{\operatorname{#1}_{\operatorname{\mathit{FDBR}}}}
\newcommand{\True}{\operatorname{\mathit{True}}}
\newcommand{\False}{\operatorname{\mathit{False}}}

\DeclareMathOperator{\discover}{discover}
\DeclareMathOperator{\discovered}{discovered}
\DeclareMathOperator{\discoveredby}{discovered by}
\DeclareMathOperator{\used}{used}
\DeclareMathOperator{\hall}{hall}

\newcommand{\relation}[1]{\operatorname{#1}_{\operatorname{\mathit{rel}}}}
\newcommand{\relationn}[3]{\operatorname{#1}_{\operatorname{\mathit{rel}}\,:\,\operatorname{\mathit{#2}} \rightarrow \operatorname{\mathit{#3}}}}
\newcommand{\relationone}[2]{\operatorname{#1}_{\operatorname{\mathit{rel}}\,:\,\operatorname{\mathit{#2}}}}

\DeclareMathOperator{\collect}{collect}

\newcommand{\FDBR}[1]{\operatorname{FDBR}({#1})}
\newcommand{\objfdbr}[1]{\operatorname{obj\_fdbr}({#1})}
\newcommand{\gatherevs}[1]{\operatorname{gather}({#1})}
\newcommand{\propfdbr}[2]{\operatorname{prop\_fdbr}({#1}, {#2})}

\DeclareMathOperator{\ent}{ent}
\DeclareMathOperator{\ev}{ev}

\newcommand{\opit}[1]{\operatorname{\mathit{#1}}}

\newcommand{\evs}{\operatorname{\mathit{evs}}}
\newcommand{\etc}{\operatorname{\mathit{etc\ldots}}}
\newcommand{\rel}{\operatorname{\mathit{rel}}}
%\newcommand{\image}{\operatorname{\mathit{image}}}
\DeclareMathOperator{\image}{image}

\newcommand{\examplespacing}{\vspace{0.5em}}

\newcommand{\examplequery}[2]{
    \texttt{#1} $\Rightarrow$ \textit{#2}
}

\newcommand{\examplequerynl}[2]{
    \texttt{#1} \\ $\Rightarrow$ \textit{#2}
}

\usepackage{environ}

\NewEnviron{examples}[1]{
\examplespacing

    \BODY

\examplespacing
}

%properties, e.g ``location'', ``implement''...

\newcommand{\property}[1]{$\opit{#1}$}

\author{Shane Peelar}
\newcommand{\dtitle}{Scalable, Efficient and Precise Natural Language Processing in the Semantic Web}
\newcommand{\defensedate}{December 7 2020}
\title{\dtitle}
\date{\defensedate}

%TODO: Explain first few years of doctoral work understanding and improving efficiency
%of parser (time complexity change?) and EAG combinators written by Richard and Rahmatullah

\begin{document}

\emergencystretch 3em %TODO: check other papers for margins TEST
%https://tex.stackexchange.com/questions/9107/how-can-i-make-my-text-never-go-over-the-right-margin-by-always-hyphenating-or-b

\setlength{\abovedisplayskip}{0pt}
\setlength{\belowdisplayskip}{12pt}
\setlength{\abovedisplayshortskip}{0pt}
\setlength{\belowdisplayshortskip}{0pt}

%\renewcommand{\gitMark}{
%	Branch: \gitBranch\,@\,\gitAbbrevHash{}
%	\textbullet{}
%	Release:\gitReln{}
%	(\gitAuthorDate)
%}

\pagenumbering{roman}

%---------

\clearpage

%TODO: change title to Scalable, Efficient and Precise Natural Language *Semantics* *for* the Semantic Web?

\thispagestyle{empty}
\begin{center}
	\vspace*{0in}

	\begin{uwinsinglespaceenv}
		\Large{\strong\dtitle}
	\end{uwinsinglespaceenv}

	\vspace{0.125in}
	%\vspace{\fill} %
	\begin{uwinonehalfspaceenv}
		by

		Shane Peelar
	\end{uwinonehalfspaceenv}
	\vspace{1.625in}

	%\normalsize
	\begin{uwinonehalfspaceenv}
	A Dissertation \\*
	Submitted to the Faculty of Graduate Studies \\*
	through the School of Computer Science \\*
	in Partial Fulfillment of the Requirements for \\*
	the Degree of Doctor of Philosophy at the \\*
	University of Windsor \\*
	\end{uwinonehalfspaceenv}

	\vspace{0.625in}
	\begin{uwinonehalfspaceenv}
	Windsor, Ontario, Canada \\

	%\vspace{0.5cm}

	\phantom{2020} \\
	%\vspace{0.5cm}

	\textcopyright \  2020 Shane Peelar
	\end{uwinonehalfspaceenv}
\end{center}

%---------


\clearpage
\thispagestyle{empty}

\begin{center}
	\begin{uwinonehalfspaceenv}
		\parbox{6in}{\centering\dtitle}
	\end{uwinonehalfspaceenv}


	%\vspace{\fill} %
	\vspace{0.35cm}
	\begin{uwinonehalfspaceenv}
		by \\*
	{Shane Peelar}
	\end{uwinonehalfspaceenv}
	%\vspace{0.7071cm}



	\vspace{0.5in}

	APPROVED BY:

	\vspace{0.5in}

	\noindent\rule{4in}{0.4pt}

	Dr. Diana Zaiu Inkpen, External Examiner

	%School of Electrical Engineering and Computer Science,
	University of Ottawa

	\vspace{0.5in}

	\noindent\rule{4in}{0.4pt}

	Dr. Richard J. Caron

	Department of Mathematics and Statistics

	\vspace{0.5in}

	\noindent\rule{4in}{0.4pt}

	Dr. Pooya Moradian Zadeh

	School of Computer Science

	\vspace{0.5in}

	\noindent\rule{4in}{0.4pt}

	Dr. Jianguo Lu

	School of Computer Science

	\vspace{0.5in}

	\noindent\rule{4in}{0.4pt}

	Dr. Richard A. Frost, Advisor

	School of Computer Science

	\vspace{0.5in}

\end{center}

\vspace*{\fill}

\hspace*{\fill}\defensedate

%\currentpdfbookmark{Copyright}{copyrightpage}%
%\noindent \textcopyright{} 2016, Shane Peelar

%\vspace{2ex}

%\noindent All Rights Reserved. Absolutely no part of this document may
%be reproduced, stored in a retrieval system, translated, in any form
%or by any means electronic, mechanical, facsimile, photocopying, or
%otherwise, without the prior written permission of the copyright
%holder.

%\vspace*{\fill}

%---------

\clearpage
% \phantomsection
\chapter*{Declaration of Co-Authorship / Previous Publication\markboth{\MakeUppercase{Declaration of Co-Authorship / Previous Publication}}{}}
\addcontentsline{toc}{chapter}{Declaration of Co-Authorship / Previous Publication}
%TODO: separate into two sections
\begin{uwindefaultspaceenv}
	\begin{enumerate}
		\item Co-Authorship Declaration

			I hereby declare that this thesis incorporates material that is result of joint research undertaken under the supervision of Dr. Richard A. Frost.  The collaboration is covered
            in Chapters 2 through 6 of the dissertation.

		\item Declaration of Previous Publication
	\end{enumerate}
\end{uwindefaultspaceenv}

\begin{uwindefaultspaceenv}


	Some of the material in this thesis is derived from the following research
	papers:

%	\begin{center}

%		\begin{tabular}{ | c | c | c | }

%			\hline
%			\parbox{0.6in}{Thesis Chapter} & \parbox{3in}{Publication title/full citation} & \parbox{1in}{Publication Status} \\

%			\hline
%			2 & \parbox{3.5in}{\fullcite{frost2018extensible}} & Published \\

%			\hline
%			3 & cell8 & Published \\
%			\hline

%			4 & cell8 & Published \\
%			\hline

%			5 & cell8 & Submitted \\
%			\hline

%		\end{tabular}

%	\end{center}

\end{uwindefaultspaceenv}


%\fullcite{frosthafiz2008}

%\fullcite{frostagboola2014}

%\fullcite{frost2014demonstration}

\begin{enumerate}
    \item \fullcite{frost2018extensible}

    \item \fullcite{frostpeelar2019}

    \item \fullcite{peelar2020compositional}

    \item A New Approach for Processing Natural-Language Queries to Semantic Web Triplestores (SUBMITTED) %TODO

    \item Accommodating Negation in an Efficient Event-Based Natural Language Query Interface to the Semantic Web (SUBMITTED) %TODO
\end{enumerate}

\begin{uwindefaultspaceenv}
	%TODO
%	Paper \cite{frosthafiz2008} (Frost, Hafiz, and Callaghan) describes a set of functional parser combinators developed by Frost and Hafiz as part of Hafiz's doctoral thesis work, which enables language processors to be built as executable specifications of fully-general attribute grammars, including ambiguous left-recursive grammars.  The processors use a polynomial time complexity top-down parsing strategy which enables a natural specification of the grammars and the associated semantic rules.  This was previously thought to be impossible, and was stated as such in many textbooks on parsing.

%	Paper \cite{frostagboola2014} (Frost, Agboola, Matthews, and Donais) describes an event based semantics developed by Dr. Frost and his research team and includes extracts from a Haskell program which demonstrated the viability of the semantics with respect to an in-program database of triples coded as part of the program.

	I am aware of the University of Windsor Senate Policy on Authorship and I certify that I have properly acknowledged the contribution of other researchers to my thesis, and have obtained written permission from each of the co-author(s) to include the above material(s) in my thesis.

	I certify that, with the above qualification, this thesis, and the research to which it refers, is the product of my own work.

	I certify that I have obtained a written permission from the copyright owner(s) to include the above
	published material(s) in my thesis. I certify that the above material describes work completed during my
	registration as graduate student at the University of Windsor.

	I declare that, to the best of my knowledge, my thesis does not infringe upon anyone's copyright nor
	violate any proprietary rights and that any ideas, techniques, quotations, or any other material from the work
	of other people included in my thesis, published or otherwise, are fully acknowledged in accordance with the
	standard referencing practices. Furthermore, to the extent that I have included copyrighted material that
	surpasses the bounds of fair dealing within the meaning of the Canada Copyright Act, I certify that I have
	obtained a written permission from the copyright owner(s) to include such material(s) in my thesis.

	I declare that this is a true copy of my thesis, including any final revisions, as approved by my thesis
	committee and the Graduate Studies office, and that this thesis has not been submitted for a higher degree to
	any other University or Institution.

\end{uwindefaultspaceenv}

%----------

\clearpage
\chapter*{Abstract\markboth{\MakeUppercase{Abstract}}{}}
\addcontentsline{toc}{chapter}{Abstract}

\begin{uwindoublespaceenv}
%TODO

\end{uwindoublespaceenv}

%----------
%
\clearpage
\chapter*{Dedication\markboth{\MakeUppercase{Dedication}}{}}
\addcontentsline{toc}{chapter}{Dedication}

\vspace*{\fill}

\begin{center}
	{\em This Thesis is dedicated to my Mémé and Pépé, Theresa and Alfred Bombardier}
\end{center}

\vspace*{\fill}

%----------

\clearpage
\chapter*{Acknowledgements\markboth{\MakeUppercase{Acknowledgements}}{}}
\addcontentsline{toc}{chapter}{Acknowledgements}

\begin{uwindefaultspaceenv}
%	I'd like to thank, in no particular order:
%
%	TODO

%	Dr. Richard A. Frost, whom I've been working with since my first semester here at the University of Windsor, for his guidance in my work
%	and the many opportunities he has provided me over the years to participate in research.  You were the first professor I had a chance to have a conversation with when I started
%	my Undergraduate degree here in 2009, and ultimately it is our conversations during the labs of 60-100 that got me seriously interested in functional programming in the first place.

%	Dr. Robert D. Kent, for believing in me and providing me with many opportunities to contribute here at the University, and generally being a very positive role model in my life.
%	It is in no small part our talks that have allowed me to refocus, get my priorities straight, and get my life back in order.

	%My External and Internal Readers, Dr. Richard J. Caron and Dr. Luis G. Rueda, for being available on such short notice to participate on my Thesis Committee.

	%David MacMillan, Bryan St. Amour and Paul Preney, for being great lab partners, being up for interesting conversations, and always being willing to lend an ear.
	%Thank you also Bryan and Paul for letting me use your LaTeX styles and giving my first draft a read.

%	All of the secretaries in the School of Computer Science for their help during my degree.  Whether it was getting me through scheduling nightmares or helping me get things done by the deadlines, your help has been incredibly invaluable to me and I sincerely thank you for your work.
%
%	My wife, Taylor, for being incredibly supportive of me both in life and in my work.  It's crazy to think about all that's happened in the last 5 years.  I'm so grateful to have had you by my side through all of it.
%
\end{uwindefaultspaceenv}

%----------
\clearpage
\phantomsection
\pdfbookmark{Table of Contents}{tableofcontentspdf}
\tableofcontents

%---------

\begin{uwindefaultspaceenv}

%TODO: order in which these appear
\clearpage
\phantomsection
\addcontentsline{toc}{chapter}{List of Tables}
\listoftables

\clearpage
\phantomsection
\addcontentsline{toc}{chapter}{List of Figures}
\listoffigures

\clearpage
\phantomsection
\addcontentsline{toc}{chapter}{List of Appendices}
\listofappendices

\clearpage
\phantomsection
\addcontentsline{toc}{chapter}{List of Abbreviations}
\listofabbreviations

%\listoftheorems[ignoreall,show={definition}]

%TODO: sort alphabetically?
%TODO: maybe list where first mentioned?
%%\addcontentsline{abc}{section}{AG - Attribute Grammar}
%%\addcontentsline{abc}{section}{CNL - Controlled Natural Language}
%%\addcontentsline{abc}{section}{CS - Compositional Semantics}
%%\addcontentsline{abc}{section}{CWA - Closed World Assumption}
%%\addcontentsline{abc}{section}{EAG - Executable Attribute Grammar}
%%\addcontentsline{abc}{section}{FDBR - Function Defined by a Binary Relation}
%%\addcontentsline{abc}{section}{LDF - Linked Data Fragments}
%%\addcontentsline{abc}{section}{MG - Montague Grammar}
%%\addcontentsline{abc}{section}{ML - Machine Learning}
%%\addcontentsline{abc}{section}{MS - Montague Semantics}
%%\addcontentsline{abc}{section}{OWA - Open World Assumption}
%%\addcontentsline{abc}{section}{NLQI - Natural Language Query Interface}
%%\addcontentsline{abc}{section}{PP - Prepositional Phrase}
%%\addcontentsline{abc}{section}{RDF - Resource Description Framework}
%%\addcontentsline{abc}{section}{SPARQL - SPARQL Protocol and RDF Query Language}

%TODO: list of figures?
%TODO: list of tables?
%TODO: list of abbreviations: page on first mentioned
%TODO: unify formatting across all pages
%TODO: link the chapters together
%TODO: complete introduction
%TODO: complete preface
%TODO: each chapter should have self-contained abbreviations

\clearpage
\pagenumbering{arabic}

\renewcommand{\refname}{Bibliography}
%\defbibheading{subbibintoc}[\refname]{\section*{#1}\markboth{#1}{#1}}

\subfile{./TeX_files/introduction}
\subfile{./TeX_files/nliwod2018conf}
\subfile{./TeX_files/webist2019conf}
\subfile{./TeX_files/icsc2020conf}
\subfile{./TeX_files/webist2019journal}
\subfile{./TeX_files/webist2020conf}


% bibliography, glossary and index would go here.

%NOTE: this is for Traditional -- we're doing Manuscript format
%\printbibliography[heading=bibintoc]

\chapter*{Appendices}
\addcontentsline{toc}{chapter}{Appendices}
\markboth{Appendices}{}

\section*{Appendix A - Source code}
\addcontentsline{toc}{section}{Appendix A - Source code listing}
\addcontentsline{apc}{appendices}{Appendix A - Source code listing}

The source code for Solarman and the XSaiga parser can be obtained online via this URL:

%TODO: up to date version
{\noindent \small \url{https://hackage.haskell.org/package/XSaiga-1.6.0.0/XSaiga-1.6.0.0.tar.gz}}

\noindent The XSaiga package for Haskell is available online at this URL:

{\noindent \small \url{https://hackage.haskell.org/package/XSaiga}}

%\lstinputlisting[language=haskell]{../solarman/XSaiga/SolarmanTriplestore.hs}
%\subsection*{SolarmanTriplestore.hs}
%\inputminted[autogobble, fontsize=\tiny, breaklines=true]{haskell}{../solarman/XSaiga/SolarmanTriplestore.hs}
%
%\subsection*{Getts.hs}
%\inputminted[autogobble, fontsize=\tiny, breaklines=true]{haskell}{../solarman/XSaiga/Getts.hs}
%
%\subsection*{TypeAg2.hs}
%\inputminted[autogobble, fontsize=\tiny, breaklines=true]{haskell}{../solarman/XSaiga/TypeAg2.hs}
%
%\subsection*{AGParser2.hs}
%\inputminted[autogobble, fontsize=\tiny, breaklines=true]{haskell}{../solarman/XSaiga/AGParser2.hs}
%
%\subsection*{Interactive.hs}
%\inputminted[autogobble, fontsize=\tiny, breaklines=true]{haskell}{../solarman/XSaiga/Interactive.hs}
%
%\subsection*{Main.hs}
%\inputminted[autogobble, fontsize=\tiny, breaklines=true]{haskell}{../solarman/Main.hs}

%\subsection*{LocalData.hs}
%\inputminted[autogobble, fontsize=\scriptsize, breaklines=true]{haskell}{../solarman/XSaiga/LocalData.hs}

\section*{Appendix B - List of Refereed Papers Relating to the Thesis}
\label{appendix:b}
\addcontentsline{toc}{section}{Appendix B - List of Refereed Papers Relating to the Thesis}
\addcontentsline{apc}{appendices}{Appendix B - List of Refereed Papers Relating to the Thesis}

Below is a list of peer-reviewed conference papers that I have authored or co-authored, which are related to this Thesis.  Electronic versions of this papers can be found: %TODO

%\cite{frosthafiz2008} \fullcite{frosthafiz2008}

%\cite{frostagboola2014} \fullcite{frostagboola2014}

\cite{donais2013system} \fullcite{donais2013system}

\cite{peelar2017windsor} \fullcite{peelar2017windsor}

\cite{peelar2018toolpath} \fullcite{peelar2018toolpath}

\cite{frost2014demonstration} \fullcite{frost2014demonstration} -- Was involved but not credited

Paper \cite{frost2014demonstration} (Frost, Donais, Matthews, Agboola, and Stewart) describes the demonstration of the Haskell program which I wrote and which forms the basis of this thesis work.  The reason that I am not listed as an author is that the paper was submitted before I officially joined the research team.  I developed the Haskell program after the paper was submitted.  The online program was the one used by Dr. Frost in the demonstration he gave at the conference this paper was presented at.

My contributions to the research project include:

%TODO: reword/shorten
\begin{itemize}
	\item Improving the efficiency of the programs which implement the event-based semantics
	\item Integrating the event-based semantics with the parser combinators to build the query processor
	\item Enhancing the existing module to access the external triplestore with efficient methods to do so, including a basic form of query fusion in the form of memoization
	\item Demonstrating a novel method of handling the word ``by'' in prepositional phrases, and extending prepositional phrases to span multiple property names
	\item Building a web interface to the query processor which includes both an English Natural Language Interface and also a safe Direct Query Interface for directly evaluating the combinators
	\item Converting the parser Hafiz wrote\cite{frosthafiz2008} to natively work with monads in Haskell, as well as the original semantics\cite{frost2014demonstration} to be monad based
	\item Maintaining the XSaiga package on {\em Hackage}\cite{XSaiga:2016}, an online repository of Haskell libraries and programs, which contains the semantics, parser, and triplestore described in this Thesis
\end{itemize}

Peer reviewed Journal papers:

\cite{peelar2019real} \fullcite{peelar2019real}

%Unrefereed papers:

%TODO: MSc thesis?

\section*{Appendix C - Copyright Releases}
\addcontentsline{toc}{section}{Appendix C - Copyright Releases}
\addcontentsline{apc}{appendices}{Appendix C - Copyright Releases}

%For NLIWoD and ICSC papers:
%© [year of original publication] IEEE. Reprinted, with permission, from [author names, paper title, IEEE publication title, and month/year of publication]

\begin{enumerate}
	\item {\Cref{chapter:nliwod2018conf} includes a refereed paper, which was published as:

	\fullcite{frost2018extensible}

	Below is the copyright information from IEEE:

	2018 IEEE. Reprinted, with permission, from Richard A. Frost and Shane Peelar, An Extensible Natural-Language Query Interface to an Event-Based Semantic Web Triplestore, October 2018.}

	\item {\Cref{chapter:webist2019conf} includes a refereed paper, which was published as:

	\fullcite{frostpeelar2019}

	Dr. Vitor Pedrosa, the chair of INSTICC, which maintains the publication for the paper, has personally sent the following e-mail to the authors regarding the copyright:


		{ \ttfamily
		Dear Shane Peelar,


		Thank you very much for your email.


		\hyphenchar\font=`\-% allowing hyphenation

		I hereby grant you the necessary authorization to add the mention
		paper to your doctoral dissertation as long as all the bibliography
		information from its publication is there too and the version
		that you use is the one published.


		Best regards,


		Vitor Pedrosa
		}

	}

	\item {\Cref{chapter:icsc2020conf} includes a refereed paper, which was published as:

		\fullcite{peelar2020compositional}

		Below is the copyright information from IEEE:

		2020 IEEE. Reprinted, with permission, from Shane Peelar and Richard A. Frost, A Compositional Semantics for a Wide-Coverage Natural-Language Query Interface to a Semantic Web Triplestore, February 2020.}

    %TODO: accepted
	\item {\Cref{chapter:webist2019journal} includes a paper that has been accepted for publication to the WEBIST Springer Journal:

    \fullcite{peelar2020webistjournal}

    As it has not yet been published, the copyright belongs to Shane Peelar and Richard A. Frost.
    }

    %TODO
	\item {\Cref{chapter:webist2020conf} includes a paper that has been submitted for publication to the WEBIST 2020 conference:

    %\fullcite{webist2020conf}

   	Dr. Vitor Pedrosa, the chair of INSTICC, which maintains the publication for the paper, has personally sent the following e-mail to the authors regarding the copyright:

    { \ttfamily
    Dear Shane Peelar,

    Thank you for your email.

    \hyphenchar\font=`\-% allowing hyphenation

    You do have permission to do that without a problem as long as the full credit is mentioned, as you already say in your email.


    Best regards,

    Vitor Pedrosa

    }

    }
\end{enumerate}








\chapter*{Vita Auctoris}
\addcontentsline{toc}{chapter}{Vita Auctoris}

Shane Peelar was born in 1990 in Windsor, Ontario.  He completed his undergraduate degree in Computer Science from the University of Windsor in 2014, graduating with Honours and specializing in Software Engineering.  He went on to complete his Masters degree in Computer Science from the University of Windsor in 2016, and is currently a candidate for the degree of Doctor of Philosophy in Computer Science.

\end{uwindefaultspaceenv}

\end{document}