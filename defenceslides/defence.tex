% !TeX spellcheck = en-US
% !TeX encoding = utf8
% !TeX program = xelatex
% !BIB program = biber

%\documentclass[logoontitle,logoonpages,tabu,supertabular,aspectratio=169]{preney-uwindsor-beamer}
\documentclass[logoontitle,tabu,supertabular,aspectratio=43]{preney-uwindsor-beamer}
%
% NOTE: "Iosevka" and "Charis SIL" fonts are set in main.sty!
%
\usepackage[
bibstyle=numeric-comp,
sorting=nyvt,
natbib=true,
mcite=true,
maxnames=100,
url=true,
isbn=false,
doi=false,
uniquename=init,
giveninits=true,
hyperref=true,
backref=true,
sortcites,
backend=biber
]{biblatex}
\usepackage{url}
\usepackage{hyperref}
\usepackage{main}

\usepackage{booktabs}
\addbibresource{../frostpaper.bib}


\newcommand{\phobos}{\mathit{phobos}}
\newcommand{\deimos}{\mathit{deimos}}
\newcommand{\every}{\mathit{every}}
\newcommand{\spins}{\mathit{spins}}
\newcommand{\spin}{\mathit{spin}}
\newcommand{\moon}{\mathit{moon}}

\newcommand{\meaningof}[1]{\lVert #1 \rVert}
\newcommand{\eventassoc}[1]{ev_{\mathrm{#1}}}
\newcommand{\entityassoc}[1]{e_{#1}}
\newcommand{\wordpred}[1]{\mathit{#1\_pred}}
\newcommand{\wordset}[1]{\mathit{#1\_set}}
\newcommand{\True}{\mathit{True}}
\newcommand{\False}{\mathit{False}}


\newcommand{\discover}{\mathit{discover}}
\newcommand{\discovered}{\mathit{discovered}}
\newcommand{\discoveredby}{\mathit{discovered\ by}}
\newcommand{\discoveredwith}{\mathit{discovered\ with}}
\newcommand{\used}{\mathit{used}}
\newcommand{\hall}{\mathit{hall}}

\newcommand{\relation}[1]{\mathit{#1\_rel}}
\newcommand{\collect}{\mathit{collect}}

\newcommand{\FDBR}[1]{\mathrm{FDBR}({#1})}
\newcommand{\objfdbr}[1]{\mathrm{obj\_fdbr}(\mathit{#1})}
\newcommand{\gatherevs}[1]{\mathrm{gather}({#1})}
\newcommand{\propfdbr}[2]{\mathrm{prop\_fdbr}(\mathit{#1, #2})}
\newcommand{\relationn}[3]{\mathit{#1\_rel_{#2 \rightarrow #3}}}

% Hack to make mintinline work in tabu tables...
% URL: https://tex.stackexchange.com/questions/372053/using-mintinline-inside-tabu/372054
\tabuDisableCommands{%
	\renewcommand\mintinline[2]{\texttt{\detokenize{#2}}}%
}

\providecommand{\emailshane}{peelar@uwindsor.ca}
\title{Scalable, Efficient and Precise Natural Language Processing in the Semantic Web}
%\subtitle{Traditional Software Development}
\date[December 17 2020]{December 17 2020}
\author[]{\small Shane Peelar \texorpdfstring{\\}{} {\scriptsize\href{mailto:\emailshane}{peelar@uwindsor.ca}}}

%\institute[WEBIST 2020 Conference]{
%	\Large WEBIST 2020 Conference\\ \vspace{1em}
%	\footnotesize Funded by NSERC of Canada
%}

\newcommand{\makesubsectionslide}{
\begin{frame}
	\centering
	\huge
	\insertsubsection
\end{frame}
}

%
% The presentation slide content follows...
%
\begin{document}
	% Set all tabu tables to have 1pt linesep by default...
	\tabulinesep=1pt

	%\partwithoutpage{\inserttitle}
	\section*{Title Page}
	\begin{frame}
	\titlepage
	\end{frame}

    \section{The Committee} %TODO: proper name for this slide
    %TODO: spruce this up with a table?
    \begin{frame}{\insertsection}
        \begin{itemize}
            \item External Examiner: Dr. Diana Zaiu Inkpen %TODO: pronunciation
            \item External Reader: Dr. Richard Caron
            \item Internal Reader: Dr. Pooya Moradian Zadeh
            \item Internal Reader: Dr. Jianguo Lu
            \item Advisor: Dr. Richard A. Frost
        \end{itemize}
    \end{frame}

	\section{Introduction}
	\begin{frame}{\insertsection}
		\begin{itemize}
			\item ...
		\end{itemize}
	\end{frame}

    %NOTE: 15-25 minutes!!!
    %TODO: sections:
    %Intro (brief)
    % - The problem!!! <-- borrow this from proposal
    % - Why important <-- borrow this from proposal
    % - Why non-trivial <-- borrow this from proposal
    % - Previous work
    % 	- Montagovian/Neo-Davidsonian...
    % 	- FLMS
    %   - UEV-FLMS
    % - Shortcomings of previous work
    % - The new idea
    %Demo (sooner? later?)
    %The five papers
    % - New idea of each paper
    % - How it contributes towards solving the problem
    %Conclusions
    % - We've shown...
    %Future work/next steps
    % - Need an empirical evaluation of the system in a real world setting
    % - QALD/DBPedia

    %SLIDES TO BORROW FROM
    % - MsC


	\section{Demonstration}
	\begin{frame}{\insertsection}
		A live demonstration of our approach is available at this URL:
		\begin{center}
			\url{http://speechweb2.cs.uwindsor.ca/solarman4/demo_sparql.html}
		\end{center}
		Some example queries that can be handled include:
        %TODO: negation examples
		\begin{itemize}
			\item \texttt{what discovered no moon in 1877}
			\item \texttt{what discovered a non moon}
			\item \texttt{allen discovered no moon at no places}
			\item \texttt{what discovered the most moons using no telescopes}
			\item \texttt{phobos and deimos were not discovered by not hall}
			\item \texttt{a person does not exist}
		\end{itemize}
	\end{frame}

	\section{The Semantic Web}
	\begin{frame}{\insertsection}
		\begin{itemize}
			\item Commonly referred to as ``Web 3.0''
			\item First coined by Tim Berners-Lee in 2001 \cite{berners2001semantic}
			\item A set of standards by the W3C for interacting with remote {\em triplestores} \cite{w3csemanticweb}
			\begin{itemize}
				\item Triples typically described as {\em subject} - {\em predicate} - {\em object}
				\item Traditionally entity based
				\item Example: \texttt{<hall> <discovered> <phobos> .}
				\item Requires {\em reification} for cross referencing
			\end{itemize}
			\item Advantages:
			\begin{itemize}
				\item Simple representation facilitates processing (RDF) \cite{w3c}
				\item No web scraping required, all machine readable
				\item Hierarchy of information an be described in OWL \cite{mcguinness2004owl}
			\end{itemize}
		\end{itemize}
	\end{frame}

	\begin{frame}{\insertsection}
	\begin{itemize}
		\item Event-based triplestores:
		\begin{itemize}
			\item The {\em subject} of a triple is the URI of an event in which something took place
			\item Data is already {\em reified}, extra information directly accessible
			\item Can add new columns and rows to a relation easily
		\end{itemize}
	\end{itemize}

	Example (N-Triples format \cite{w3cntriples}, full URI omitted):

	\hspace{5em}\texttt{<event1045> <subject>\phantom{qqq}<hall> .}

	\hspace{5em}\texttt{<event1045> <object>\phantom{qqqq}<phobos> .}

	\hspace{5em}\texttt{<event1045> <type>\phantom{qqqqqq}<discover\_ ev> .}

	\hspace{5em}\texttt{<event1045> <year>\phantom{qqqqqq}<1877> .}

	\hspace{5em}\texttt{<event1045> <location>\phantom{qq}<us\_naval\_observatory> .}

	\hspace{5em}\texttt{<event1045> <implement>\phantom{q}<refractor\_telescope\_1> .}

	%TODO: send Richard paper, Tiffany Chien, Jugal Kalita Adversarial Analysis of NLIS.  We may be able to handle some of these.  IDEA: P => Q, P generates data (assumes it) and Q consumes it via query
	\end{frame}

	\section{Natural Language is inherently difficult}

	\begin{frame}{\insertsection}
		\begin{itemize}
			\item NL is inherently syntactically ambiguous:
			\begin{itemize}
				\item \texttt{discover a moon that spins in 1877}
				\item \texttt{discover (a (moon that spins)) [in 1877]}
				\item \texttt{discover (a (moon that (spins [in 1877])))}
			\end{itemize}
			\item Semantic ambiguity also exists: consider the word ``depart''
			\item Users may not enter grammatically correct sentences
			\item Users may misspell words
			\item Users may not clearly state what they mean
            \item \textbf{How to handle negation?}
			\item The situation does not look good for arbitrary user input!
		\end{itemize}
	\end{frame}

	\section{Constraining the Problem}
	\begin{frame}{\insertsection}
		\begin{itemize}
			\item But for NLQIs, the situation isn't as bad as it seems
			\item The set of possible inputs are constrained to Q\&A-type sentences
			\item A full understanding of the underlying NL is not needed
			\item NLQIs may be {\em wide} or {\em narrow} in scope
			\item Machine Learning is effective at dealing with {\em wide}-scoped interfaces
			\item But how can we handle highly specific queries, such as: \texttt{how many moons that orbit a red planet were discovered by nicholson}?
			\item The goal of this research is to produce a framework for NLQIs that can be used with expert systems and highly domain specific databases for precise {\em narrow} queries
		\end{itemize}
	\end{frame}


%TODO: do I really want to try to categorize this stuff?
%	\begin{frame}{\insertsection}
%		\begin{itemize}
%			\item The most popular approaches when dealing with NL involve using Machine Learning (ML)
%			\item Many libraries and frameworks available to facilitate this
%			\item ML is effective at dealing with a wide scope of queries %TODO
%			\item But ML approaches have trouble when things get specific:
%			\begin{itemize}
%				\item \texttt{which person discovered a moon that spins in 1877 with a shiny telescope}
%			\end{itemize}
%			\item Various metrics to evaluate ML approaches: precision, accuracy, recall.. %TODO
%			\item ML approaches try to classify the intent behind a query and present relevant information %TODO, need a citation for that
%		\end{itemize}
%	\end{frame}

	\section{Compositional Semantics}
	\begin{frame}{\insertsection}
		\begin{itemize}
			\item One way to handle NL is to use a {\em Compositional Semantics} (CS)
			\item The idea: each word in the query has a corresponding mathematical {\em denotation} that allows the query to be treated as a mathematical expression
			\begin{itemize}
				\item \texttt{who discovered a moon} $\Rightarrow$ \texttt{who discovered (a moon)}
				\item \texttt{hall discovered a vacuumous moon in 1877} \\ $\Rightarrow$ \texttt{hall (discovered (a (vacuumous moon)) [in 1877])}
			\end{itemize}
			\item First described by Richard Montague \cite{Dowty:wall}
			\begin{itemize}
				\item ``I reject the contention that an important theoretical difference exists between formal and natural languages.'' (Montague, 1970)
				\item The denotations are formally described in the Typed Lambda Calculus
			\end{itemize}
		\end{itemize}
	\end{frame}

	\begin{frame}{\insertsection}
		\begin{itemize}
			\item When dealing with structured queries, CS has a few advantages:
			\begin{itemize}
				\item Extensible -- new denotations able to be added without modifying existing ones
				\item Syntactic categories can be type checked: ``\texttt{hall}'' and ``\texttt{a person}'' have the same type and can be substituted
				\item Expressive -- small set of rules define an infinite language
				\item Can be used directly with an Executable Attribute Grammar \cite{frost1989constructing} to ``execute'' NL as if it were code
				\item Proofs directly in the language itself
				\item The result of the query is as correct as the underlying database
			\end{itemize}
		\end{itemize}
	\end{frame}

	\section{The FDBR}
	\begin{frame}{\insertsection}
		\begin{itemize}
			\item The {\em Function Defined by a Binary Relation}
			\item The idea: convert an arbitrary BR into a function by grouping items in the domain together with the sets of entities they map to in the codomain
			\item For example:
			\begin{equation*}
			\relationn{discover}{subject}{object} = \{(\entityassoc{hall}, \entityassoc{phobos}),(\entityassoc{hall},\entityassoc{deimos}),\ldots\}
			\end{equation*}
			\begin{equation*}
			\FDBR{\relationn{discover}{subject}{object}} = \big\{(\entityassoc{hall}, \{\entityassoc{phobos}, \entityassoc{deimos}\}), \ldots\big\}
			\end{equation*}
			\item First used by Frost et. al in 1989 \cite{frost1989constructing} to provide a denotation for binary transitive verbs in MS
		\end{itemize}
	\end{frame}

	\begin{frame}{\insertsection}
		\begin{itemize}
			\item Much later, it was realized by Peelar and Frost that it could be used to answer many kinds of queries in NL
			\item In 2016, Peelar showed that it is possible to unify the treatment of complex English constructs (UEV-FLMS) using the FDBR \cite{peelar2016accommodating}
			\begin{itemize}
				\item A query can be thought of as a filter over the underlying database
			\end{itemize}
			\item In particular, it could be used to accommodate chained Prepositional Phrases, which was a common critique of MS approaches
			\item We have recently shown it can be used to handle superlative phrases as well
			\begin{itemize}
				\item To our knowledge, no other CS-based systems have been able to do this
			\end{itemize}
		\end{itemize}
	\end{frame}

	\section{Our Approach}
	\begin{frame}{\insertsection}
		\begin{itemize}
			\item The scoping for the denotations is determined by parsing
			\begin{itemize}
				\item Ambiguous queries will have multiple valid parses
				\item Semantic ambiguity is permitted -- a word may have multiple possible denotations
			\end{itemize}
			\item Let users decide what they mean, rather than try to guess it
			\begin{itemize}
				\item Expose the possible meanings/parses through the interface
			\end{itemize}
			\item Use an {\em event-based} view of information \cite{frost:eswcposter2014}
			\begin{itemize}
				\item Results are {\em auditable} since the resulting FDBR will contain the events that justify their inclusion
			\end{itemize}
			\item Constrain the problem to expert systems for domain-specific problems
			\begin{itemize}
				\item Let the {\em wide}-scoped interfaces delegate to {\em narrow} systems where appropriate
			\end{itemize}
		\end{itemize}
	\end{frame}

    \section{Negation}
    \begin{frame}{\insertsection}
        \begin{itemize}
            \item RDF is based on the Open World Assumption, which can be summarized as: \linebreak
            ``\textit{The absence of evidence cannot be construed as being evidence of absence.}''
            \item This makes negation tricky to work with:
            \begin{itemize}
                \item Obtaining the cardinality of a set may not be possible
                \item Or it may be so large that it is not practical to work with
            \end{itemize}
            \item But the Closed World Assumption may still hold for domain specific applications
            \item \textbf{How can we support use cases where the CWA holds?}
        \end{itemize}
    \end{frame}

	\section{Our Contribution}
    \begin{frame}{\insertsection}
        \begin{itemize}
            \item The new contribution of this paper is that we can now accommodate negation in the semantics where the CWA holds.
            \item The new idea: The notion of the complement of an FDBR
            \begin{itemize}
                \item Track the cardinality of FDBRs and the cardinality of the set of all entities during a query
                \item Extend the FDBR operations (union, intersect) to support \texttt{ComplementFDBR}s
                \item Requires only the cardinality, not the actual complement, to evaluate many queries
                \item ``drop-in'' denotations for ``not'', ``no'', and ``non''
                \item Denotation for transitive verbs supporting superlatives, prepositional phrases, and negation
            \end{itemize}
            \item Integrates seamlessly with memoization framework described by Peelar in \cite{peelar2020webistjournal}
        \end{itemize}
    \end{frame}

    \begin{frame}{\insertsection}
    \begin{itemize}
        \item This makes it possible, where the Closed World Assumption holds, to evaluate Natural Language queries to triplestores having the words ``\texttt{no}'', ``\texttt{not}'', and ``\texttt{non-}''
        \item Our approach also supports the negation of term-phrases, such as ``\texttt{hall}'', which was missing in previous work
        \item Double negation: ``\texttt{not not hall discovered not not phobos}''
        \item ``drop-in'' support: remove the denotations to revert to the OWA
        \item SQUALL\cite{ferre2013squall}: ``\textit{Which author of Paper42 has not affiliation Salford University?}''
        \item Our approach: ``\textit{Which author of Paper42 is not affiliated with Salford University?}''
    \end{itemize}
    \end{frame}
%	\begin{frame}{\insertsection}
%		\begin{itemize}
%			\item The new contribution of this paper is that we can now handle $N$-ary transitive verbs in a compositional manner
%			\begin{itemize}
%				\item The new idea: Convert $N$-ary relations to binary relations, then apply the existing FDBR machinery
%			\end{itemize}
%			\item This makes it possible to query from subject to object, subject to implement, object to year, etc. $N(N -1) = N^2 - N$ choices.
%			\item We also note that there is a direct correspondence between event-based triplestores and relational databases:
%			\begin{itemize}
%				\item Event identifier corresponds to the Primary Key
%				\item Event type corresponds to the relation or table
%				\item Event roles correspond to the columns of the relation
%			\end{itemize}
%		\end{itemize}
%	\setlength{\tabcolsep}{0.4em} % for the horizontal padding
%	\renewcommand{\arraystretch}{1}% for the vertical padding
%		\vspace{-1.5em}
%		\begin{table}
%			\caption{\footnotesize\normalfont\scshape\\The "Discover'' Relation}
%			\centering
%			\vspace{-0.25cm}
%			\begin{tabular}{  c c c c c  }
%				\hline \hline
%				subject & object & date & implement & location \\
%				\hline
%				\dots & \dots & \dots & \dots & \dots \\
%				hall & phobos & 1877 & refractor\_telescope\_1 & us\_naval\_observatory \\
%				\dots & \dots & \dots & \dots & \dots \\
%				\hline \hline
%			\end{tabular}
%			\vspace{-1em}
%			\label{tab:discover}
%		\end{table}
%	\end{frame}


%	\begin{frame}{\insertsection}
%		\begin{itemize}
%
%			\item This allows us to handle complex linguistic constructs such as chained prepositional phrases
%			%\item This would be far too limiting -- in SPARQL for example, we would not be able to handle chained prepositional phrases
%			\item Instead, gather smaller queries and batch these into larger queries where possible
%			\item The individual denotations are directly derived using set theory, and the answer is as correct as the data
%		\end{itemize}
%	\end{frame}

	\section{Future Work}
	\begin{frame}{\insertsection}
		\begin{itemize}
			\item Non-event based triplestores (Timbr.ai)
            \item Applications
            \item Working directly within the browser with WebAssembly*
            \begin{itemize}
                \item Prototype: \url{https://speechweb2.cs.uwindsor.ca/solarman-wasm/}
            \end{itemize}
		\end{itemize}
	\end{frame}

	\section{Thank you!}
	\begin{frame}{\insertsection}
		\begin{center}
			\huge Thank you for attending!  Questions/comments?
		\end{center}
	\end{frame}

%	\section{Appendix}
%	\subsection{Overview}
%	\begin{frame}{\insertsection: \insertsubsection}
%	%\begin{block}{Overview}
%	%	Semantics $\leftrightarrow$ Event bridge $\leftrightarrow \{\textit{ML}, \textit{Schemas}\} \leftrightarrow$ Triplestore
%	%\end{block}
%
%	\begin{itemize}
%		\item Must discuss event-based and non-event-based triplestores
%		\item Must discuss reification methods and why they are needed
%		\item Provide motivating examples adding facts to existing info
%	\end{itemize}
%
%
%	\end{frame}

%	\subsection{Previous Work}
%	\begin{frame}{\insertsection: \insertsubsection}
%	Two dominant approaches in NL:
%	\begin{itemize}
%		\item Denotational Semantics: good for \textit{narrow} and \textit{precise} queries (precise answers but requires precise semantics)
%		\item Machine Learning: good for \textit{broad} queries (it finds relevant sections of web pages, but it is only a guess)
%	\end{itemize}
%	Both approaches attempt to translate NL queries into SPARQL\cite{sparql}
%	It is non-trivial because:
%	\begin{itemize}
%		\item Many people are working on the problem
%		\item It is not solved yet
%	\end{itemize}
%	\end{frame}

%	\subsection{Why is it useful? Examples}
%	\begin{frame}{\insertsection: \insertsubsection}
%	\begin{itemize}
%		\item Querying the periodic table (want narrow)
%		\item Querying a wide variety of information (want broad)
%		\item Querying a legal database (want narrow)
%	\end{itemize}
%	\end{frame}

%	\subsection{Hardware acceleration}
%	\begin{frame}{\insertsection: \insertsubsection}
%	I need a section on hardware acceleration, including FPGAs, CUDA, and OpenCL
%	\end{frame}

%	\subsection{Planned publications}
%	\begin{frame}{\insertsection: \insertsubsection}
%	\begin{itemize}
%		\item Spoke in Paris 45 mins seminar
%		\item Spoke during Colloquium Seminars last year for OpenCL
%	\end{itemize}
%	\end{frame}

%	\subsection{FDBR and Goals}
%	\begin{frame}{\insertsection: \insertsubsection}
%	\begin{itemize}
%		\item DBPedia
%		%TODO: must mention SQUALL
%		\item \textbf{No one can do superlatives or compartives} -- quote from 2014 Ferre p.37
%		\item We can do:
%		\begin{itemize}
%			\item Superlative
%			\item Negation
%			\item Preposition
%			\item Comparative
%		\end{itemize}
%		\item FDBR ($n^2 - n$) -- this to be mentioned when going over NLIWoD
%		\item Transitive verbs explicit denotation
%		\item Discuss shortcomings with direct SPARQL translation.
%		%\item Ferre discusses precision in conclusions.
%		%\item Research how SQUALL translates prepositional phrases and the limitations
%	\end{itemize}
%	\end{frame}
%
%	\subsection{ISWC}
%	\begin{frame}{\insertsection: \insertsubsection}
%	\begin{itemize}
%		%\item NLIWoD is a specific workshop, was absorbed into ISWC
%		\item The paper deals with the main shortcomings of MS
%		\item Reawakening interest in MS
%		%\item Why did MS fall out of favour? go over this?  was it mentioned in Squall 2014?
%		%\item Richard will send citation for Ferre Squall 2014
%	\end{itemize}
%	\end{frame}



	%\part{Introduction}
	%==========================================================================

	%\part{References}
	%\sectionwithouttoc{Questions}
	%\begin{frame}{\insertsection}
	%  Questions\par
	%\end{frame}

	\section{References}
	\begin{frame}[allowframebreaks]{References}
	\printbibliography
	\end{frame}

	\section{Appendix}
	\begin{frame}{\insertsection}
		\begin{itemize}
			\item So, why aren't CS used more?  Several reasons:
			\begin{itemize}
				\item Complex linguistic constructs have been traditionally hard to handle
				\item The CS as described by Montague is computationally intractable
				\item Montague's denotation for transitive verbs is very complicated
				\item Not robust with respect to user input
				\item Need to translate the NL query to some kind of structured query language \cite{hoffart2013yago2}
				\begin{itemize}
					\item Underlying query language needs to be complex enough to support NL
				\end{itemize}
			\end{itemize}
		\end{itemize}
	\end{frame}

	\begin{frame}{\insertsection}
		\begin{itemize}
			\item There have been major advancements to address these shortcomings:
			\begin{itemize}
			\item In 1989, Frost et. al described FLMS, a tractable version of Montague's semantics, along with a denotation for $2$-ary transitive verbs \cite{frost1989constructing}
			\item In 2014, it was shown that direct translation to an underlying query language is not required (EV-FLMS) \cite{frost2014denotational}
			\item In 2016, Peelar showed that chained prepositional phrases can be handled in a CS (Masters Thesis) \cite{peelar2016accommodating}
			%\item In YEAR, Frost and Peelar showed that superlatives can also be handled %TODO
			\item In 2020, we showed that $N$-ary transitive verbs can be handled compositionally \cite{peelar2020compositional}
            \item In 2020, we showed it is possible to memoize the semantics to improve the asymptotic complexity significantly \cite{peelar2020webistjournal}
            \item In 2020, we now show it is possible to accommodate negation efficiently as well
			\end{itemize}
		\end{itemize}
	\end{frame}


	\begin{frame}{\insertsection}
		\begin{itemize}
			\item We have shown that it is not necessary to directly translate the query to an underlying query language \cite{frost2014demonstration}
			\begin{itemize}
				\item The denotation itself contains a small set of queries for the information they need
				\item Batch these smaller queries together
			\end{itemize}
			\item Three simple ``query'' functions used, can be implemented in any query language
			\begin{itemize}
				\item Implementable directly in SPARQL, Triple Pattern Fragments (LDF), and even SQL
			\end{itemize}
		\end{itemize}
	\end{frame}

	\begin{frame}{\insertsection}
		\begin{itemize}
			\item Our implementation is written entirely in Haskell
			\item It is able to be used even on low power devices, such as consumer routers
			\begin{itemize}
				\item \url{https://solarman.inbetweennames.net}
			\end{itemize}
			\item The FDBRs are able to be generated offline or on the fly
			\item The semantics are completely memoized for efficient computation
			\item The code is completely open source, available on Hackage \cite{xsaiga}
            \item Now can run entirely within the web browser:
            \begin{itemize}
                \item \url{https://speechweb2.cs.uwindsor.ca/solarman-wasm/}
            \end{itemize}
		\end{itemize}
	\end{frame}

%   \appendix
%   \section{An Appendix Title}
%   \begin{frame}{\insertsection}
%     This is the first section.
%   \end{frame}
%
%   \section{Another Appendix Title}
%   \begin{frame}{\insertsection}
%     This is the second section.
%   \end{frame}
\end{document}
